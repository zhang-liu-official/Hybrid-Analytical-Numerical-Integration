%---------DO NOT EDIT THIS INDENTED SECTION
	% Preamble
	\documentclass[11pt,reqno,oneside,a4paper]{article}
	\usepackage[a4paper,includeheadfoot,left=35mm,right=35mm,top=00mm,bottom=30mm,headheight=40mm]{geometry} %sets up the margins
	\setcounter{section}{-1}
	\input{texHead-Proof-Standard} % Use the standard texHead for this module. You should not edit this file.
	\input{texHead-Proof-Theorems} % Use the standard theorem definitions for this module. You should not edit this file.
	%---The following code defines the title, author, and date of the document.
	\title{Notes on Langer Paper}
	\author{Zhang Liu}
	\date{\today}   % Using \today automatically updates to the document's build date
%----------------------------------
%---------IF YOU WANT TO DEFINE YOUR OWN MACROS, YOU CAN DO SO FROM HERE ...

%---------... TO HERE
\begin{document}
\maketitle
\thispagestyle{fancy}

\section{Introduction}

	First, we define an exponential sum to be a type of function in the form
	\begin{equation} 
	\tag{1} \Phi(z) = \sum_{j=0}^{n}A_{j}(z) e^{c_{j}z}, 
	\end{equation}
	where the coefficients $A_{j}(z)$ and the exponents $c_{j}$ are constants, $c_{j}\in \RR.$ The exponential sum will be the subject of discussion. 
	
	This paper aims to consider the distribution of zeros of exponential sums, from forms of less to forms of greater generality. It is relevant to our project because these results help us 'eliminate' terms in the exponential sums that can be considered as zero, i.e., the terms that do not contribute as much to the final value of the exponential sum. This will in turn significantly speed up the process of numerically evaluating the integration of exponential sums. 
	
	For the purpose of our project, we will only focus on the following cases: (1) Constant Coefficients and Real Commensurable Exponents. (2) Constant Coefficients and General Real Exponents. (3) Coefficients Asymptotically Constant. (4) Collinear Complex Exponents. (5) General Complex Constants. (Note that the original theorems 5 and 6 are less relevant to our project because the polynomials are assumed to be of the same degree in the scope of our project.)
	
	
\section{Constant Coefficients and Real Commensurable Exponents.}

The paper starts by discussing the simplest case: when the coefficients are constants ($A_j(z) \equiv a_j$) and the exponents are commensurable real numbers. Two real numbers are commensurable if their ratio is rational, for example, $2\pi$ and $4\pi$ are commensurable because $\frac{2\pi}{4\pi} = \frac{1}{2} \in \QQ$. 

This case is the simplest because then the problem of the distribution of zeros essentially becomes an algebraic one: to simply arrange the terms to suppose that the exponents occur in the order of increasing algebraic magnitude and to assume without loss of generality that the first exponent $c_0$ is zero (to remove any exponential factor that does not really matter, i.e., that does not affect the number or distribution of the zeros of the function as a whole).

In this case, the sum is of the form:
\begin{equation} 
\tag {3} \Phi(z) = \sum_{j=0}^{n}a_{j}(e^{\alpha z})^{p_j}, p_0 = 0,
\end{equation}

for example, let $\alpha = \pi, n = 4, p_j = j$

$$\Phi(z) = a_0(e^z)^0 + a_1(e^{\pi z})^1 + a_2(e^{\pi z})^2 + a_3(e^{\pi z})^3 + a_4(e^{\pi z})^4,$$

which is a polynomial of degree $p_n$ in the quantity $e^{\alpha z}$.

\begin{thm}
	If in the exponential sum (1) the coefficients are constants and the exponents are real and commensurable, then the distribution of zeros is given explicitly by the formula:
	
	$$z = \frac{1}{\alpha}\{2m \pi i + \log\xi_{j}\},$$
	$$(j = 1,2,\cdots, p_n), (m = 0, \pm1, \pm2,\cdots).$$
	
	In this distribution the number of zeros which lie between two lines $y=y_1$ and $y=y_2$, is restricted by the relations (7), i.e., 
	\begin{equation} 
	\tag{7} -n + \frac{c_n}{2\pi}(y_2-y_1) \leq n(R) \leq n + \frac{c_n}{2\pi}(y_2-y_1).
	\end{equation}
\end{thm}

Explanation of Theorem 1:  

When $z = \frac{1}{\alpha}\{2m \pi i + \log\xi_{j}\},$ the function vanishes for such values of $z$. Using the example for the form (3), 

\begin{thm}
	If the coefficients $a_j$ are real and the zeros of the polynomial 
	$$P(\xi)= \sum_{j=0}^{n}a_{j}\xi^{j}$$
	all lie within the unit circle about $\xi=0$, then the zeros of the corresponding trigonometric sums are all real and simple, where the trigonometric sums are:
	$$\Phi_{c}(z) = \sum_{j=0}^{n}a_{j}\cos jz$$
	$$\Phi_{s}(z) = \sum_{j=1}^{n}a_{j}\sin jz.$$	
	
	Each of these sums has precisely $2n$ zeros on the interval $0 \leq z < 2\pi$ and the zeros of either sum alternate with those of the other. (By a theorem of Kakya the hypothesis is fulfilled if $0 \leq a_0 < a_1 < \cdots < a_n$.)	
\end{thm}

\section{Constant Coefficients and General Real Exponents.}
	Under the case of ``Constant Coefficients and General Real Exponents," the sum $\Phi(z)$ is expressed by the formula:
	\begin{equation} 
	\tag {10} \Phi(z) = \sum_{j=0}^{n}a_{j}e^{c_{j}z}, c_0 = 0.
	\end{equation}
\begin{thm}
	If in the exponential sum (10) the coefficients are constants and the exponents are real, then the zeros of the sum all lie within a strip (5), i.e., 
	\begin{equation}
	\tag{5} \abs x < K (z = x + iy),
	\end{equation}
	and in any portion of this strip the number of zeros is limited by relation (7), i.e., 
	\begin{equation} 
	\tag{7} -n + \frac{c_n}{2\pi}(y_2-y_1) \leq n(R) \leq n + \frac{c_n}{2\pi}(y_2-y_1).
	\end{equation}
	
	When $z$ is uniformly bounded from the zeros of $\Phi(z)$, then $\abs{\Phi(z)}$ is uniformly bounded from zero. 
\end{thm}

\section{Coefficients Asymptotically Constant.}
Under the case of ``Coefficients Asymptotically Constant," the form assumed for the sum (1) is:
	\begin{equation} 
	\tag {14} \Phi(z) = \sum_{j=0}^{n}\{a_{j} + \epsilon(z)\}e^{c_{j}z}, a_0a_n \neq 0.
	\end{equation}
\begin{thm}
	If the function $\Phi(z)$ (or a determination of it) is of the form (14), then in the region $\abs z > M$ the distribution of zeros of $\Phi(z)$ (or of the branch of $\Phi(z)$ in question) may be described as in Theorem 3. The zeros are asymptotically represented by those of the related sum (15), i.e.,
	\begin{equation}
	\tag{15} \Phi_1(z) = \sum_{j=0}^{n} a_{j}e^{c_{j}z}.
	\end{equation}
\end{thm}

\section{Collinear Complex Exponents.}

The previous cases of sum (1) concern primarily with the structure of the coefficient functions $A_{j}(z)$. The exponents have been assumed to be real. But the distribution of zeros of the sum having complex exponents is also determinable.

\begin{thm}
	If the  exponents $c_j$ in the exponential sum (1) are collinear complex constants, the distribution of zeros of $\Phi(z)$ is obtainable from the theorem previously enunciated by substituting in the role of the axis of reals the line containing the points $\bar c_j$ conjugate to the exponents $c_j$.
\end{thm}

\section{General Complex Constants.}

\begin{thm}
	If in the sum (1) the exponents are any complex constants, the zeros of $\Phi(z)$ are confined for $\abs z >M$ to a finite number of strips each of asymptotically constant width. These strips are associated in groups with the exterior normals to the sides of the polygon described in the text, and approach parallelism with the respective normals. Within each group of strips the distribution of zeros may be described as in the previously stated theorems, the role of the axis of reals being transferred to the respective side of the polygon.
\end{thm}

\end{document}



