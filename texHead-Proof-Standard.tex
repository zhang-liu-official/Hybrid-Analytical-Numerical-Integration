%%%%%%%%%%%%%%%%%%%%%%%%%%%%%%%%%%%%%%%%%%%%%%%%%%%%%%%%%%%%%%%%%%%%%%%%%%%%%%%%
%
% This file contains some standard modifications to basic LaTeX2e and
% the article documentclass. DO NOT EDIT THIS FILE, but do look through
% and make use of the shorthands defined herein.
%
%%%%%%%%%%%%%%%%%%%%%%%%%%%%%%%%%%%%%%%%%%%%%%%%%%%%%%%%%%%%%%%%%%%%%%%%%%%%%%%%

% Standard packages
\usepackage{amssymb,amsmath,amsthm}
\usepackage{xcolor,graphicx}
\usepackage{verbatim}
\usepackage{hyperref}
% Layout of headers & footers
\usepackage{titling}
\usepackage{fancyhdr}
\pagestyle{fancy} \lhead{{\theauthor}} \chead{} \rhead{} \lfoot{} \cfoot{\thepage} \rfoot{}

% Hyphenation
\hyphenation{non-zero}

% Instructor's email address
\newcommand{\InstEmail}{dave.smith@yale-nus.edu.sg}

%% Mathmode shortcuts
% Number sets
\newcommand{\NN}{\mathbb N}              % The set of naturals
\newcommand{\NNzero}{\NN^0}              % The set of naturals including zero
\newcommand{\NNone}{\NN}                 % The set of naturals excluding zero
\newcommand{\ZZ}{\mathbb Z}              % The set of integers
\newcommand{\QQ}{\mathbb Q}              % The set of rationals
\newcommand{\RR}{\mathbb R}              % The set of reals
\newcommand{\CC}{\mathbb C}              % The set of complex numbers
\newcommand{\KK}{\mathbb K}              % An arbitrary field
% Modern typesetting for the real and imaginary parts of a complex number
\renewcommand{\Re}{\operatorname*{Re}} \renewcommand{\Im}{\operatorname*{Im}}
% Upright d for derivatives
\newcommand{\D}{\ensuremath{\,\mathrm{d}}}
% Make epsilons look more different from the element symbol
\renewcommand{\epsilon}{\varepsilon}
% Always use slanted forms of \leq, \geq
\renewcommand{\geq}{\geqslant}
\renewcommand{\leq}{\leqslant}
% Shorthand for some relations
\newcommand{\po}{\preceq}
\newcommand{\rel}{{\mathcal R}} \newcommand{\rels}{\mathbin{\scriptstyle{\mathcal R}}}
% Shorthand for "if and only if" symbol
\newcommand{\Iff}{\ensuremath{\Leftrightarrow}}
% Make bold symbols for vectors
\providecommand{\BVec}[1]{\mathbf{#1}}
% Barred forms of \oplus and \otimes to represent the descents of these binary operators
\newcommand{\oplusbar}{\mathbin{\ooalign{$\hidewidth\overline{\oplus}\hidewidth$\cr$\phantom{\oplus}$}}} \newcommand{\otimesbar}{\mathbin{\ooalign{$\hidewidth\overline{\otimes}\hidewidth$\cr$\phantom{\otimes}$}}}
% Mathematical operators used in Proof
\DeclareMathOperator{\sgn}{sgn}          % The signum of a real number
\DeclareMathOperator{\power}{\mathcal{P}} % The power set of a set
\DeclareMathOperator{\Id}{Id}            % The identity function
\DeclareMathOperator{\Fun}{Fun}          % The set of functions from one set to another
\DeclareMathOperator{\Perm}{Perm}        % The set of permutations on a set
\DeclareMathOperator{\GCD}{GCD}          % The greatest common divisor of two integers
\newcommand{\abs}[1]{\left\lvert#1\right\rvert} % The absolute value of a real number or modulus of a complex number, with automatically scaling delimiters
