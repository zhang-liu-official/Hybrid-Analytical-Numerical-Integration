%---------DO NOT EDIT THIS INDENTED SECTION
	% Preamble
	\documentclass[11pt,reqno,oneside,a4paper]{article}
	\usepackage[a4paper,includeheadfoot,left=35mm,right=35mm,top=00mm,bottom=30mm,headheight=40mm]{geometry} %sets up the margins
	%%%%%%%%%%%%%%%%%%%%%%%%%%%%%%%%%%%%%%%%%%%%%%%%%%%%%%%%%%%%%%%%%%%%%%%%%%%%%%%%
%
% This file contains some standard modifications to basic LaTeX2e and
% the article documentclass. DO NOT EDIT THIS FILE, but do look through
% and make use of the shorthands defined herein.
%
%%%%%%%%%%%%%%%%%%%%%%%%%%%%%%%%%%%%%%%%%%%%%%%%%%%%%%%%%%%%%%%%%%%%%%%%%%%%%%%%

% Standard packages
\usepackage{amssymb,amsmath,amsthm}
\usepackage{xcolor,graphicx}
\usepackage{verbatim}
\usepackage{hyperref}
% Layout of headers & footers
\usepackage{titling}
\usepackage{fancyhdr}
\pagestyle{fancy} \lhead{{\theauthor}} \chead{} \rhead{} \lfoot{} \cfoot{\thepage} \rfoot{}

% Hyphenation
\hyphenation{non-zero}

% Instructor's email address
\newcommand{\InstEmail}{dave.smith@yale-nus.edu.sg}

%% Mathmode shortcuts
% Number sets
\newcommand{\NN}{\mathbb N}              % The set of naturals
\newcommand{\NNzero}{\NN^0}              % The set of naturals including zero
\newcommand{\NNone}{\NN}                 % The set of naturals excluding zero
\newcommand{\ZZ}{\mathbb Z}              % The set of integers
\newcommand{\QQ}{\mathbb Q}              % The set of rationals
\newcommand{\RR}{\mathbb R}              % The set of reals
\newcommand{\CC}{\mathbb C}              % The set of complex numbers
\newcommand{\KK}{\mathbb K}              % An arbitrary field
% Modern typesetting for the real and imaginary parts of a complex number
\renewcommand{\Re}{\operatorname*{Re}} \renewcommand{\Im}{\operatorname*{Im}}
% Upright d for derivatives
\newcommand{\D}{\ensuremath{\,\mathrm{d}}}
% Make epsilons look more different from the element symbol
\renewcommand{\epsilon}{\varepsilon}
% Always use slanted forms of \leq, \geq
\renewcommand{\geq}{\geqslant}
\renewcommand{\leq}{\leqslant}
% Shorthand for some relations
\newcommand{\po}{\preceq}
\newcommand{\rel}{{\mathcal R}} \newcommand{\rels}{\mathbin{\scriptstyle{\mathcal R}}}
% Shorthand for "if and only if" symbol
\newcommand{\Iff}{\ensuremath{\Leftrightarrow}}
% Make bold symbols for vectors
\providecommand{\BVec}[1]{\mathbf{#1}}
% Barred forms of \oplus and \otimes to represent the descents of these binary operators
\newcommand{\oplusbar}{\mathbin{\ooalign{$\hidewidth\overline{\oplus}\hidewidth$\cr$\phantom{\oplus}$}}} \newcommand{\otimesbar}{\mathbin{\ooalign{$\hidewidth\overline{\otimes}\hidewidth$\cr$\phantom{\otimes}$}}}
% Mathematical operators used in Proof
\DeclareMathOperator{\sgn}{sgn}          % The signum of a real number
\DeclareMathOperator{\power}{\mathcal{P}} % The power set of a set
\DeclareMathOperator{\Id}{Id}            % The identity function
\DeclareMathOperator{\Fun}{Fun}          % The set of functions from one set to another
\DeclareMathOperator{\Perm}{Perm}        % The set of permutations on a set
\DeclareMathOperator{\GCD}{GCD}          % The greatest common divisor of two integers
\newcommand{\abs}[1]{\left\lvert#1\right\rvert} % The absolute value of a real number or modulus of a complex number, with automatically scaling delimiters
 % Use the standard texHead for this module. You should not edit this file.
	%%%%%%%%%%%%%%%%%%%%%%%%%%%%%%%%%%%%%%%%%%%%%%%%%%%%%%%%%%%%%%%%%%%%%%%%%%%%%%%%
%
% This file contains some standard modifications to basic LaTeX2e and
% the article documentclass. DO NOT EDIT THIS FILE, but do look through
% and make use of the shorthands defined herein.
%
% This file should be input only after texHead-Proof-Standard.tex,
% specifically after the line
% \usepackage{amssymb,amsmath,amsthm}
% of that file.
%
%%%%%%%%%%%%%%%%%%%%%%%%%%%%%%%%%%%%%%%%%%%%%%%%%%%%%%%%%%%%%%%%%%%%%%%%%%%%%%%%

% Theorem definitions in the amsthm standard
\newtheorem{thm}{Theorem}
\newtheorem{lem}[thm]{Lemma}
\newtheorem{sublem}[thm]{Sublemma}
\newtheorem{prop}[thm]{Proposition}
\newtheorem{cor}[thm]{Corollary}
\newtheorem{conc}[thm]{Conclusion}
\newtheorem{conj}[thm]{Conjecture}
\theoremstyle{definition}
\newtheorem{defn}[thm]{Definition}
\newtheorem{cond}[thm]{Condition}
\newtheorem{asm}[thm]{Assumption}
\newtheorem{ntn}[thm]{Notation}
\newtheorem{prob}[thm]{Problem}
\theoremstyle{remark}
\newtheorem{rmk}[thm]{Remark}
\newtheorem{eg}[thm]{Example}
\newtheorem*{hint}{Hint}
 % Use the standard theorem definitions for this module. You should not edit this file.
	%---The following code defines the title, author, and date of the document.
	\title{Notes on Langer 1973}
	\author{Zhang Liu}
	\date{\today}   % Using \today automatically updates to the document's build date
%----------------------------------
%---------IF YOU WANT TO DEFINE YOUR OWN MACROS, YOU CAN DO SO FROM HERE ...

%---------... TO HERE
\begin{document}
\maketitle
\thispagestyle{fancy}

\section{Introduction}
	An exponential sum is a type of function in the form
	\begin{equation} 
	\tag{1} \Phi(z) = \sum_{j=0}^{n}A_{j}(z) e^{c_{j}z}, 
	\end{equation}
	where $A_{j}(z)$ and $c_{j}$ are constants, $c_{j}\in \RR.$
	
	\bigskip
	
	The function (1) can be expressed in a form
	\begin{equation} 
	\tag{2} \Phi(z - z_0) = z \int_{c_0}^{c_n}\phi(t)e^{tz}dt. 
	\end{equation}
	
	The integral of the type here involved with less specialized function $\phi(t)$ represents a generalization of certain sums of type (1).
	
	
\section{Constant Coefficients and Real Commensurable Exponents.}

\begin{thm}
	If in the exponential sum (1) the coefficients are constants and the exponents are real and commensurable, then the distribution of zeros is given explicitly by the formula:
	
	$$z = \frac{1}{\alpha}\{2m \pi i + \log\xi_{j}\},$$
	$$(j = 1,2,\cdots, p_n, m = 0, \pm1, \pm2,\cdots).$$
	
	In this distribution the number of zeros which lie between two lines $y=y_1$ and $y=y_2$, is restricted by the relation (7), i.e., 
	
	$$-n + \frac{c_n}{2\pi}(y_2-y_1) \leq n(R) \leq n + \frac{c_n}{2\pi}(y_2-y_1).$$
\end{thm}

\begin{thm}
	If the coefficients $a_j$ are real and the zeros of the polynomial 
	$$P(\xi)= \sum_{j=0}^{n}a_{j}\xi^{j}$$
	all lie within the unit circle about $\xi=0$, then the zeros of the corresponding trigonometric sums are all real and simple, where the trigonometric sums are:
	$$\Phi_{c}(z) = \sum_{j=0}^{n}a_{j}\cos jz$$
	$$\Phi_{s}(z) = \sum_{j=1}^{n}a_{j}\sin jz.$$	
	
	Each of these sumes has precisely $2n$ zeros on the interval $0 \leq z < 2\pi$ and the zeros of either sum alternate with those of the other. (By a theorem of Kakya the hypothesis is fulfilled if $0 \leq 1_0 < a_1 < \cdots < a_n$.)	
\end{thm}

\section{Constant Coefficients and General Real Exponents.}
	Under the case of "Constant Coefficients and General Real Exponents," the sum $\Phi(z)$ is expressed by the formula:
	\begin{equation} 
	\tag {10} \Phi(z) = \sum_{j=0}^{n}a_{j}e^{c_{j}z}, c_0 = 0.
	\end{equation}
\begin{thm}
	If in the exponential sum (10) the coefficients are constants and the exponents are real, then the zeros of the sum all lie within a strip (5), i.e., 
	
	$$\abs(x) < K (z = x + iy),$$
	and in any portion of this strip the number of zeros is limited by relation (7), i.e., 
	
	$$-n + \frac{c_n}{2\pi}(y_2-y_1) \leq n(R) \leq n + \frac{c_n}{2\pi}(y_2-y_1).$$
	
	When $z$ is uniformly bounded from the zeros of $\Phi(z)$, then $\abs\Phi(z)$ is uniformly bounded from zero. 
\end{thm}

\section{Coefficients Asymptotically Constant.}
Under the case of "Coefficients Asymptotically Constant," the form assumed for the sum (1) is:
	\begin{equation} 
	\tag {14} \Phi(z) = \sum_{j=0}^{n}a_{j + \epsilon(z)}e^{c_{j}z}, a_0a_n \neq 0.
	\end{equation}
\begin{thm}
	If the function $\Phi(z)$ (or a determination of it) is of the form (14), then in the region $\abs(z) > M$ the distribution of zeros of $\Phi(z)$ (or of the branch of $\Phi(z)$ in question) may be described as in Theorem 3. The zeros are asymptotically represented by those of the related sum (15), i.e.,
	$$\Phi_1(z) = \sum_{j=0}^{n} a_{j}e^{c_{j}z}.$$
\end{thm}

\section{Coefficients which are Asymptotically Power Functions.}
\section{The Values $v_j$ and $c_j$ Proportional.}

\begin{thm}
	If in the exponential sum (1) the coefficients are of the form (16) with values $v_j$ proportional to the exponents $c_j$, and all terms are ordinary terms, then the zeros of the sum are asymptotically located within a logarithmic curvilinear strip bounded by curves of the form (18), i.e., 
\end{thm}
\end{document}



